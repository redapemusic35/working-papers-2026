% Options for packages loaded elsewhere
% Options for packages loaded elsewhere
\PassOptionsToPackage{unicode}{hyperref}
\PassOptionsToPackage{hyphens}{url}
\PassOptionsToPackage{dvipsnames,svgnames,x11names}{xcolor}
%
\documentclass[
  letterpaper,
  DIV=11,
  numbers=noendperiod]{scrreprt}
\usepackage{xcolor}
\usepackage{amsmath,amssymb}
\setcounter{secnumdepth}{-\maxdimen} % remove section numbering
\usepackage{iftex}
\ifPDFTeX
  \usepackage[T1]{fontenc}
  \usepackage[utf8]{inputenc}
  \usepackage{textcomp} % provide euro and other symbols
\else % if luatex or xetex
  \usepackage{unicode-math} % this also loads fontspec
  \defaultfontfeatures{Scale=MatchLowercase}
  \defaultfontfeatures[\rmfamily]{Ligatures=TeX,Scale=1}
\fi
\usepackage{lmodern}
\ifPDFTeX\else
  % xetex/luatex font selection
\fi
% Use upquote if available, for straight quotes in verbatim environments
\IfFileExists{upquote.sty}{\usepackage{upquote}}{}
\IfFileExists{microtype.sty}{% use microtype if available
  \usepackage[]{microtype}
  \UseMicrotypeSet[protrusion]{basicmath} % disable protrusion for tt fonts
}{}
\makeatletter
\@ifundefined{KOMAClassName}{% if non-KOMA class
  \IfFileExists{parskip.sty}{%
    \usepackage{parskip}
  }{% else
    \setlength{\parindent}{0pt}
    \setlength{\parskip}{6pt plus 2pt minus 1pt}}
}{% if KOMA class
  \KOMAoptions{parskip=half}}
\makeatother
% Make \paragraph and \subparagraph free-standing
\makeatletter
\ifx\paragraph\undefined\else
  \let\oldparagraph\paragraph
  \renewcommand{\paragraph}{
    \@ifstar
      \xxxParagraphStar
      \xxxParagraphNoStar
  }
  \newcommand{\xxxParagraphStar}[1]{\oldparagraph*{#1}\mbox{}}
  \newcommand{\xxxParagraphNoStar}[1]{\oldparagraph{#1}\mbox{}}
\fi
\ifx\subparagraph\undefined\else
  \let\oldsubparagraph\subparagraph
  \renewcommand{\subparagraph}{
    \@ifstar
      \xxxSubParagraphStar
      \xxxSubParagraphNoStar
  }
  \newcommand{\xxxSubParagraphStar}[1]{\oldsubparagraph*{#1}\mbox{}}
  \newcommand{\xxxSubParagraphNoStar}[1]{\oldsubparagraph{#1}\mbox{}}
\fi
\makeatother


\usepackage{longtable,booktabs,array}
\usepackage{calc} % for calculating minipage widths
% Correct order of tables after \paragraph or \subparagraph
\usepackage{etoolbox}
\makeatletter
\patchcmd\longtable{\par}{\if@noskipsec\mbox{}\fi\par}{}{}
\makeatother
% Allow footnotes in longtable head/foot
\IfFileExists{footnotehyper.sty}{\usepackage{footnotehyper}}{\usepackage{footnote}}
\makesavenoteenv{longtable}
\usepackage{graphicx}
\makeatletter
\newsavebox\pandoc@box
\newcommand*\pandocbounded[1]{% scales image to fit in text height/width
  \sbox\pandoc@box{#1}%
  \Gscale@div\@tempa{\textheight}{\dimexpr\ht\pandoc@box+\dp\pandoc@box\relax}%
  \Gscale@div\@tempb{\linewidth}{\wd\pandoc@box}%
  \ifdim\@tempb\p@<\@tempa\p@\let\@tempa\@tempb\fi% select the smaller of both
  \ifdim\@tempa\p@<\p@\scalebox{\@tempa}{\usebox\pandoc@box}%
  \else\usebox{\pandoc@box}%
  \fi%
}
% Set default figure placement to htbp
\def\fps@figure{htbp}
\makeatother





\setlength{\emergencystretch}{3em} % prevent overfull lines

\providecommand{\tightlist}{%
  \setlength{\itemsep}{0pt}\setlength{\parskip}{0pt}}



 


\KOMAoption{captions}{tableheading}
\makeatletter
\@ifpackageloaded{tcolorbox}{}{\usepackage[skins,breakable]{tcolorbox}}
\@ifpackageloaded{fontawesome5}{}{\usepackage{fontawesome5}}
\definecolor{quarto-callout-color}{HTML}{909090}
\definecolor{quarto-callout-note-color}{HTML}{0758E5}
\definecolor{quarto-callout-important-color}{HTML}{CC1914}
\definecolor{quarto-callout-warning-color}{HTML}{EB9113}
\definecolor{quarto-callout-tip-color}{HTML}{00A047}
\definecolor{quarto-callout-caution-color}{HTML}{FC5300}
\definecolor{quarto-callout-color-frame}{HTML}{acacac}
\definecolor{quarto-callout-note-color-frame}{HTML}{4582ec}
\definecolor{quarto-callout-important-color-frame}{HTML}{d9534f}
\definecolor{quarto-callout-warning-color-frame}{HTML}{f0ad4e}
\definecolor{quarto-callout-tip-color-frame}{HTML}{02b875}
\definecolor{quarto-callout-caution-color-frame}{HTML}{fd7e14}
\makeatother
\makeatletter
\@ifpackageloaded{caption}{}{\usepackage{caption}}
\AtBeginDocument{%
\ifdefined\contentsname
  \renewcommand*\contentsname{Table of contents}
\else
  \newcommand\contentsname{Table of contents}
\fi
\ifdefined\listfigurename
  \renewcommand*\listfigurename{List of Figures}
\else
  \newcommand\listfigurename{List of Figures}
\fi
\ifdefined\listtablename
  \renewcommand*\listtablename{List of Tables}
\else
  \newcommand\listtablename{List of Tables}
\fi
\ifdefined\figurename
  \renewcommand*\figurename{Figure}
\else
  \newcommand\figurename{Figure}
\fi
\ifdefined\tablename
  \renewcommand*\tablename{Table}
\else
  \newcommand\tablename{Table}
\fi
}
\@ifpackageloaded{float}{}{\usepackage{float}}
\floatstyle{ruled}
\@ifundefined{c@chapter}{\newfloat{codelisting}{h}{lop}}{\newfloat{codelisting}{h}{lop}[chapter]}
\floatname{codelisting}{Listing}
\newcommand*\listoflistings{\listof{codelisting}{List of Listings}}
\makeatother
\makeatletter
\makeatother
\makeatletter
\@ifpackageloaded{caption}{}{\usepackage{caption}}
\@ifpackageloaded{subcaption}{}{\usepackage{subcaption}}
\makeatother
\usepackage{bookmark}
\IfFileExists{xurl.sty}{\usepackage{xurl}}{} % add URL line breaks if available
\urlstyle{same}
\hypersetup{
  colorlinks=true,
  linkcolor={blue},
  filecolor={Maroon},
  citecolor={Blue},
  urlcolor={Blue},
  pdfcreator={LaTeX via pandoc}}


\author{}
\date{}
\begin{document}


\section{Two Classes of Evil in Medieval
Bestiaries}\label{two-classes-of-evil-in-medieval-bestiaries}

\begin{itemize}
\tightlist
\item
  \textbf{Deceivers}: Camouflage, imitation, seduction

  \begin{itemize}
  \tightlist
  \item
    Chameleon: camouflage
  \item
    Manticore: accurate imitations
  \item
    Sirens: seductive skills
  \end{itemize}
\item
  \textbf{Fear and Death Generators}
\item
  Purpose: Educate illiterate via art for \textbf{orthodoxy} and proper
  behavior (Piñero Moral 2021)
\end{itemize}

Medieval taxonomy mirrors modern epistemic harms: misinformation, false
narratives, bullshit.

\section{Deception in Bestiaries → Modern Epistemic
Harms}\label{deception-in-bestiaries-modern-epistemic-harms}

\begin{itemize}
\tightlist
\item
  Medieval worries: Not just epistemic, but close to today

  \begin{itemize}
  \tightlist
  \item
    \textbf{Misinformation} (chameleon-like camouflage)
  \item
    \textbf{False/inauthentic narratives} (manticore imitation)
  \item
    \textbf{Bullshit} (siren seduction)
  \end{itemize}
\end{itemize}

\begin{quote}
Beasts symbolize \textbf{evil} to promote right action.
\end{quote}

\begin{center}\rule{0.5\linewidth}{0.5pt}\end{center}

\section{Hermeneutic Injustice (Fricker
2007)}\label{hermeneutic-injustice-fricker-2007}

\begin{itemize}
\tightlist
\item
  \textbf{Definition}: Gap in collective interpretive resources

  \begin{itemize}
  \tightlist
  \item
    Hinders making sense of one's own experiences
  \end{itemize}
\item
  \textbf{Link to Deceptive Evil}

  \begin{itemize}
  \tightlist
  \item
    Restricts hermeneutic participation in social life
  \item
    Historically against women, but not exclusively
  \end{itemize}
\item
  \textbf{Manticore} as artistic representation of this evil
\end{itemize}

\begin{quote}
Evil is ``direct and intuitive by the victim'' (Russell 2016, 1).
\end{quote}

Hermeneutic injustice deceives like the manticore.

\section{Hermeneutic Codes in
Bestiaries}\label{hermeneutic-codes-in-bestiaries}

\begin{quote}
Beastiaries composed of \textbf{hermeneutic codes} mapped to levels:
anatomy → ethical-philosophical → moral → theological (Voisenet 2020,
321--25). Words and images intertwined (Kay 2017).
\end{quote}

\section{Types of Codes}\label{types-of-codes}

\begin{longtable}[]{@{}ll@{}}
\toprule\noalign{}
Level & Focus \\
\midrule\noalign{}
\endhead
\bottomrule\noalign{}
\endlastfoot
Mere Anatomy & Physical description \\
Ethical-Philosophical & Moral reasoning \\
Moral Perspectives & Behavior norms \\
Theological & God vs.~Devil \\
\end{longtable}

\section{Art and Complex Reality}\label{art-and-complex-reality}

\begin{quote}
``Commitment with reality undeniable, yet so complex\ldots{} limits
blurred between what \textbf{is} and \textbf{is not}, what we
\textbf{believe} and \textbf{know}\ldots{}'' (Cronin 1941).
\end{quote}

\begin{itemize}
\tightlist
\item
  \textbf{Conflict}: God vs.~Devil

  \begin{itemize}
  \tightlist
  \item
    Jesus' death/resurrection: Devil's action → God's victory
  \end{itemize}
\item
  Art configures \textbf{human condition}: experience, knowledge,
  beliefs
\end{itemize}

\begin{center}\rule{0.5\linewidth}{0.5pt}\end{center}

\begin{tcolorbox}[enhanced jigsaw, leftrule=.75mm, coltitle=black, bottomtitle=1mm, titlerule=0mm, colback=white, breakable, colbacktitle=quarto-callout-important-color!10!white, opacityback=0, arc=.35mm, colframe=quarto-callout-important-color-frame, toprule=.15mm, title=\textcolor{quarto-callout-important-color}{\faExclamation}\hspace{0.5em}{Important}, rightrule=.15mm, left=2mm, bottomrule=.15mm, opacitybacktitle=0.6, toptitle=1mm]

Medieval \emph{ars} \textgreater{} mere norms; profound lecture on
reality.

\end{tcolorbox}

\section{Origins of the Manticore}\label{origins-of-the-manticore}

\begin{itemize}
\tightlist
\item
  \textbf{Persian literature} (5th--4th BC, Ctesias)

  \begin{itemize}
  \tightlist
  \item
    Red skin, lion-sized, \textbf{human-like face}, 3 rows of teeth,
    \textbf{scorpion tail}
  \item
    Eats men; imitates voices (Pliny)
  \end{itemize}
\item
  \textbf{Evolution}:
\end{itemize}

\begin{center}\rule{0.5\linewidth}{0.5pt}\end{center}

\begin{longtable}[]{@{}ll@{}}
\toprule\noalign{}
Source & Key Traits \\
\midrule\noalign{}
\endhead
\bottomrule\noalign{}
\endlastfoot
Aristotle & Rough as lion, grey eyes, sting darts spines \\
Pliny & Fixed gaze, continuous teeth, imitates speech \\
Ctesias & Azure eyes, blood color, flute/trumpet voice \\
\end{longtable}

\begin{quote}
Became symbol of \textbf{evil, death, hell} → Christ's power over them.
\end{quote}

\begin{center}\rule{0.5\linewidth}{0.5pt}\end{center}

\section{Familiar vs.~Unfamiliar
Imagery}\label{familiar-vs.-unfamiliar-imagery}

\begin{itemize}
\tightlist
\item
  \textbf{Familiar}: Animals (lion, scorpion)
\item
  \textbf{Unfamiliar}: Human face, triple teeth → \textbf{Immaterial
  evil}
\item
  \textbf{Why?} Easier to intuit abstract concepts: pain, death, good
  vs.~evil
\item
  \textbf{Hermeneutic Tool}: Confront evil's attributes (seduces,
  deceives, harms)
\end{itemize}

\begin{center}\rule{0.5\linewidth}{0.5pt}\end{center}

\textbf{Bodleian Library Examples} (placeholders): - Fol. 009v: Early
folio - Fol. 025r: Classic manticore (MS. Bodl. 764) - Fol. 022v:
Variant (MS. Ashmole 1511)

\begin{figure}[H]

{\centering \pandocbounded{\includegraphics[keepaspectratio]{thesis-outline_files/mediabag/.pdf}}

}

\caption{Manticore Placeholder}

\end{figure}%

\section{Emotions, Cognition, and
Identity}\label{emotions-cognition-and-identity}

\textbf{Gut Reactions (Prinz)}: Emotions affect cognition

\begin{longtable}[]{@{}ll@{}}
\toprule\noalign{}
Emotion Type & Cognitive Effect \\
\midrule\noalign{}
\endhead
\bottomrule\noalign{}
\endlastfoot
Positive & Stereotypes, creative reasoning (Isen et al.~1987) \\
Negative & Narrow focus, threats (Ohman et al.~2001) \\
Congruent & Better recall (Bower 1981) \\
\end{longtable}

\begin{center}\rule{0.5\linewidth}{0.5pt}\end{center}

\begin{itemize}
\tightlist
\item
  \textbf{Identity Links}:

  \begin{itemize}
  \tightlist
  \item
    Narrative Identity
  \item
    Personal Identity (response to emotions)
  \end{itemize}
\end{itemize}

Emotions shape self-understanding.

\section{Black Authenticity \& Controlling Images (Laybourn
2018)}\label{black-authenticity-controlling-images-laybourn-2018}

\begin{itemize}
\tightlist
\item
  \textbf{Criticism}: Identity commodified as ``realness''

  \begin{itemize}
  \tightlist
  \item
    Authentic? Male, female, ethnic\ldots{}
  \end{itemize}
\item
  \textbf{Hermeneutic Injustice}:

  \begin{itemize}
  \tightlist
  \item
    Popular media images accepted as representative
  \item
    Blurs harm vs.~authenticity
  \end{itemize}
\item
  \textbf{Core Question}: Narratives → Personal Identity? \textgreater{}
  ``We are defined by stories we accept about ourselves''
  (controversial).
\end{itemize}

\begin{center}\rule{0.5\linewidth}{0.5pt}\end{center}

\section{Wicked: Modern Analogy for
Othering}\label{wicked-modern-analogy-for-othering}

\begin{itemize}
\tightlist
\item
  \textbf{Elphaba} (Cynthia Erivo): Green skin → scapegoat

  \begin{itemize}
  \tightlist
  \item
    Leans into ``wicked'' persona
  \end{itemize}
\item
  \textbf{Trope}: Water bucket (racist bathing myth)
\item
  \textbf{Outcome}: ``Dies'' (hides), flees as eternal outsider with
  Fiyero
\item
  \textbf{Link}: Hermeneutic gap → false self-beliefs
\end{itemize}

\begin{quote}
Medieval manticore \textbf{lends credibility} to epistemic injustice
claims.
\end{quote}

\section{Medieval vs.~Modern Art
Functions}\label{medieval-vs.-modern-art-functions}

\begin{longtable}[]{@{}
  >{\raggedright\arraybackslash}p{(\linewidth - 4\tabcolsep) * \real{0.2394}}
  >{\raggedright\arraybackslash}p{(\linewidth - 4\tabcolsep) * \real{0.3803}}
  >{\raggedright\arraybackslash}p{(\linewidth - 4\tabcolsep) * \real{0.3803}}@{}}
\toprule\noalign{}
\begin{minipage}[b]{\linewidth}\raggedright
Aspect
\end{minipage} & \begin{minipage}[b]{\linewidth}\raggedright
Medieval
\end{minipage} & \begin{minipage}[b]{\linewidth}\raggedright
Modern (Gilmore 2020)
\end{minipage} \\
\midrule\noalign{}
\endhead
\bottomrule\noalign{}
\endlastfoot
\textbf{Purpose} & Orthodoxy, behavior motivation & Pleasure,
entertainment \\
\textbf{Outcome} & Desirable actions/dispositions & Absorption,
revenue \\
\textbf{Example} & Manticore → intuit evil & Advertising campaigns \\
\end{longtable}

\begin{center}\rule{0.5\linewidth}{0.5pt}\end{center}

\begin{itemize}
\tightlist
\item
  \textbf{Medieval}: ``Right reason'' (Aquinas, Summa Theologica I-II,
  Q.57)
\item
  Good cognitive use → virtuous actions/objects
\end{itemize}

Gaskell's \emph{Mary Barton} → social safety net (Harrison 2011).

\section{Moral Reality in Medieval Art (Piñero Moral
2021)}\label{moral-reality-in-medieval-art-piuxf1ero-moral-2021}

\begin{quote}
Art represents \textbf{complex reality}, esp.~moral (De Bruyne 1959).
\end{quote}

\begin{itemize}
\tightlist
\item
  \textbf{Manticore}: Evil via projection (obscure/contested reality)
\item
  \textbf{Not} mere facts/history → \textbf{wished beliefs} for
  orthodoxy
\item
  \textbf{Key}: Motivate behavior for good outcomes
\end{itemize}

\textbf{Conclusion}: Bestiaries combat \textbf{deceptive evil} like
hermeneutic injustice.

\begin{center}\rule{0.5\linewidth}{0.5pt}\end{center}

\section{References}\label{references}

\begin{itemize}
\tightlist
\item
  Fricker, Miranda. 2007. \emph{Epistemic Injustice}. Oxford University
  Press.
\item
  Gilmore, Jonathan. 2020. \emph{Apt Imaginings}. Oxford University
  Press.
\item
  Harrison, Mary-Catherine. 2011. ``Narrative Relationships\ldots{}''
  \emph{Poetics Today} 32(2).
\item
  Laybourn, Wendy M. 2018. ``Cost of Being `Real'\ldots{}'' \emph{Ethnic
  and Racial Studies} 41(11).
\item
  Piñero Moral, Ricardo. 2021. ``Aesthetics of Evil\ldots{}''
  \emph{Religions} 12(957).
\item
  Russell, Jeffrey Burton. 2016. \emph{The Prince of Darkness}. Cornell
  University Press.
\item
  Others: Voisenet 2020, Kay 2017, Cronin 1941, etc.
\end{itemize}

Full citations in text. Customize images/links as needed.




\end{document}
